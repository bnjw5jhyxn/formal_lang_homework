\def\abs#1{\vert#1\vert}
\def\desc#1{\langle#1\rangle}
\def\floor#1{\lfloor#1\rfloor}
\def\reducesto{\le _{\hbox{m,poly}}}
\centerline{Steve Hsu\hfill homework 5}
\item{6.14.}

Let $A$ and $B$ be languages.
Take $J = \{(x, 0) : x \in A\} \cup \{(y, 1) : y \in B\}$.
To decide $A$ on input $x$,
call the oracle for $J$ on $(x, 0)$.
Similarly, to decide $B$ on output $y$,
call the oracle for $J$ on $(y, 1)$.
\bigskip
\item{6.19.}

We can describe an oracle Turing machine
with an oracle for $A_{TM}$ with a finite string.
There are therefore countably many such oracle Turing machines.
Since there are uncountably many languages,
there are not enough oracle Turing machines to recognize all languages.
\bigskip
\item{7.5.} no

Let $\ell _x$ and $\ell _y$ be an arbitrary assignment of $x$ and $y$.
Notice that there is a clause $(\overline{\ell _x} \vee \overline{\ell _y})$,
so the formula is not satisfied.
\bigskip
\item{7.6.}

Let $L_1$ and $L_2$ be languages in $P$.
Then there are deciders $M_1$ and $M_2$
that decide $L_1$ and $L_2$ in $O(n^c)$ and $O(n^d)$ time, respectively.


In order to decide $L_1 \cup L_2$ on input $x$, first run $M_1$ on $x$.
If it accepts, then accept $x$.
Otherwise, run $M_2$ on $x$ and output its answer.
This procedure takes $O(\abs x ^c + \abs x ^d)$ time,
so $L_1 \cup L_2 \in P$.

In order to decide $L_1 \cdot L_2$ on input $x = x_1 x_2 \cdots x_n$,
for each integer $k$ such that $0 \le k \le n$,
take $a_k = x_1 x_2 \cdots x_k$ and $b_k = x_{k+1} x_{k+2} \cdots x_n$.
Run $M_1$ on $a_k$ and $M_2$ on $b_k$.
If both accept, then accept $x$.
If none of the iterations accept, then reject $x$.
This procedure takes $O(n(n^c + n^d))$,
so $L_1 \cdot L_2 \in P$.

In order to decide $\overline{L_1}$,
run $M_1$ on $x$ and output the negation of its answer.
This procedure takes $O(\abs x ^c)$ time,
so $\overline{L_1} \in P$.
\bigskip
\item{7.7.}

Let $L_1$ and $L_2$ be languages in $NP$.
Then there are deciders $M_1$ and $M_2$ such that
$x \in L_1$ if and only if there exists $y \in \Sigma ^\star$
such that $M_1(x, y)$ accepts
and $x \in L_2$ if and only if there exists $y \in \Sigma ^\star$
such that $M_2(x, y)$ accepts.
Additionally, $M_1$ runs in $O(n^c)$ time and $M_2$ runs in $O(n^d)$ time.

In order to verify membership of $x$ in $L_1 \cup L_2$ with witness $y$,
run $M_1$ and $M_2$ on $(x, y)$.
If at least one accepts, then accept $(x, y)$.
Otherwise, reject $(x, y)$.
This procedure takes $O(\abs{(x,y)}^c + \abs{(x,y)}^d)$ time.

In order to verify membership of $x = x_1 \cdots x_n$ in $L_1 \cdot L_2$
with witness $y$, if $y$ is not of the form $(y_1, y_2)$
where $y_1, y_2 \in \Sigma ^\star$, then reject $(x, y)$.
For each integer $k$ such that $0 \le k \le n$,
take $a_k = x_1 \cdots x_k$ and $b_k = x_{k+1} \cdots x_n$.
Run $M_1$ on $(a_k, y_1)$ and $M_2$ on $(b_k, y_2)$.
If both accept, then accept $(x, y)$.
If none of the iterations accept, then reject $x$.
This procedure takes $O(\abs{(x,y)} (\abs{(x,y)}^c + \abs{(x,y)}^d)$ time.
\bigskip
\item{7.9.}

Let the vertex set of $G$ be $v_1, v_2, \ldots, v_n$.
There are $O(n^3)$ ordered triples of integers $(i,j,k)$
such that $1 \le i < j < k \le n$.
For each such triple, check if $(i,j)$, $(i,k)$, and $(j,k)$
are in the edge set of $G$.
If not all of them are, then reject $\desc G$.
If none of the iterations reject, then accept $\desc G$.
Since it takes linear time to search for a pair of edges on a tape,
this procedure takes $O(\abs{\desc G} ^4)$ time.
\bigskip
\item{7.13.}

Let $r(a, n)$ be the smallest nonnegative integer $k$
such that $a \equiv k \pmod n$.
Notice that we can compute $r$ in quadratic time.
Let $f(t,a,b,p) = t a^b \pmod p$.
We can compute this function recursively in polynomial time.
If $b = 0$, then return $t$.
If $b > 0$, then let $b' = \floor{b/2}$ and
$x = n$ if $b$ is odd and $1$ if $b$ is even.
Return $f(r(tx, p), r(a^2, p), b', p)$.
To decide $MODEXP$ on input $\desc{a,b,c,p}$,
if $f(1,a,b,p) \equiv c \pmod p$, then accept.
Otherwise, reject.
Notice that we make $O(\log b)$ recursive calls to $f$,
and we need quadratic time in each step for the multiplications
and reductions $\bmod p$,
so this procedure takes $O(\abs{\desc{a,b,c,p}}^3)$ time.
\bigskip
\item{7.14.}

Notice that it takes $O(\log k)$ space
to describe a permutation on $\{1,\ldots,k\}$.
Let $f(h,q,t) = h \circ q^t$.
We can compute this function recursively in polynomial time.
If $t = 0$, then return $h$.
If $t > 0$, then let $t' = \floor{t/2}$ and
$x = q$ if $t$ is odd and $x = I$ if $t$ is even.
Return $f(hx, q \circ q, t')$.
To decide $PERM$-$POWER$ on input $\desc{p,q,t}$,
accept if $f(I,q,t) = p$.
Otherwise, reject.
Notice that we make $O(\log t)$ recursive calls to $f$
and that we can compute compositions of permutations in polynomial time.
Therefore, this procedure takes polynomial time.
\bigskip
\item{7.17.}

The proof fails because representing in unary results in an exponential
increase in the size of the representations of the numbers in the input,
possibly causing reductions to take exponential time to compute the mapping.

\bigskip
\item{7.18.}

Let $A \in P$ with $A \ne \emptyset$ and $A \ne \Sigma ^\star$.
Since $P \subset NP$, we have that $A \in NP$.
Since $A \ne \emptyset$, there exists $a \in A$.
Since $A \ne \Sigma ^\star$, there exists $b \in \overline A$.
Let $L \in NP$.
Since $P = NP$, $L \in P$.
To reduce $L$ to $A$ on input $x$, decide if $x \in L$.
This step takes polynomial time since $L \in P$.
If $x \in L$, then output $a$.
Otherwise, output $b$.
Therefore, $L \reducesto A$, so $A$ is $NP$-hard.
Since $A \in NP$ and $A$ is $NP$-hard, $A$ is $NP$-complete, as desired.
\bigskip
\item{7.20.}

By contraposition, the claim is equivalent to the implication:
if $P = NP$, then $PATH$ is $NP$-complete.
Assume that $P = NP$.
Since $PATH \in P$, $PATH \ne emptyset$, and $PATH \ne \Sigma ^\star$,
by exercise (7.18), $PATH$ is $NP$-complete, as desired.
\bigskip
\item{7.22.}

Since we can verify membership in $DOUBLE$-$SAT$ with two distinct
satisfying assignments, $DOUBLE$-$SAT \in NP$.
We will show bi reduction from $SAT$ that $DOUBLE$-$SAT$ is $NP$-hard.
Given a boolean formula $\phi$,
let $\phi'$ be a boolean formula identical to $\phi$
except that every variable $x_i$ that appears in $\phi$
is renamed to a distinct variable $y_i$ that does not appear in $\phi$.
Let $f(\desc\phi) = \desc{\phi \wedge \phi' \wedge
\bigvee _{x_i \in \phi} (x_i \oplus y_i)}$,
where $\oplus$ is the exclusive or operator.
The $\phi \wedge \phi'$ part ensures that both assignments are satisfying
and the rest ensures that the assignments are distinct.
\bigskip
\item{7.34.}

We will show that $D$ is $NP$-hard by reduction from $SUBSET$-$SUM$.
Let $(a_1,a_2,\ldots,a_n,b)$ be an input to $SUMSET$-$SUM$.
Output the polynomial
$((\sum _{i=1} ^n a_i x_i) - b)^2 + \sum _{i=1} ^n (x_i (x_i - 1))^2$,
where $x_1, \ldots, x_n$ are real numbers.
Notice that all of the component polynomials are squared and are therefore nonnegative.
Therefore, any root must make all of these components zero.
The $x_i(x_i - 1)$ parts ensure that each $x_i$ is either $1$ or $0$
and the $\sum a_i x_i - b$ part ensures that any root specifies a subset
of $\{a_i\}$ that sums to $b$.
\bigskip
\item{7.39.}

Let $L$ be the set of pairs $(n,k)$ such that
$n$ has a nontrivial factor at most $k$.
Notice that if $n$ is composite,
then we can binary search for the smallest prime factor of $n$
in $\log n$ steps.
Additionally, since each prime factor of $n$ is at least $2$,
$n$ has at most $\log n$ prime factors, counting repetitions.
We can therefore factor integers with $\log ^2 n$ calls
to a decider for $L$.
If $P = NP$, then such a decider runs in polynomial time
and we can factor integers quickly.
\bye
