\input miniltx
\input graphicx.sty
\input tikz.tex
\usetikzlibrary{automata,positioning}
\centerline{Steve Hsu\hfill homework 1}
\item{6.} a.

\tikzpicture[auto]
\node[state,initial,initial text=] (q_0) {$q_0$};
\node[state] (q_1) [right=of q_0] {$q_1$};
\node[state,accepting] (q_2) [right=of q_1] {$q_2$};
\path[->]
(q_0) edge node {$1$} (q_1)
(q_1) edge [bend left=10] node {$0$} (q_2)
      edge [loop above] node {$1$} ()
(q_2) edge [loop above] node {$0$} ()
      edge [bend left=10] node {$1$} (q_1);
\endtikzpicture

\item{} b.

\tikzpicture[auto]
\node[state,initial,initial text=] (q_0) {$q_0$};
\node[state] (q_1) [right=of q_0] {$q_1$};
\node[state] (q_2) [right=of q_1] {$q_2$};
\node[state,accepting] (q_3) [right=of q_2] {$q_3$};
\path[->]
(q_0) edge [loop above] node {$0$} ()
      edge node {$1$} (q_1)
(q_1) edge [loop above] node {$0$} ()
      edge node {$1$} (q_2)
(q_2) edge [loop above] node {$0$} ()
      edge node {$1$} (q_3)
(q_3) edge [loop right] node {$0,1$} ();
\endtikzpicture

\item{} g.

\tikzpicture[auto]
\node[state,initial,accepting,initial text=] (q_0) {$q_0$};
\node[state,accepting] (q_1) [right=of q_0] {$q_1$};
\node[state,accepting] (q_2) [right=of q_1] {$q_2$};
\node[state,accepting] (q_3) [right=of q_2] {$q_3$};
\node[state,accepting] (q_4) [right=of q_3] {$q_4$};
\node[state,accepting] (q_5) [right=of q_4] {$q_5$};
\path[->]
(q_0) edge node {$0,1$} (q_1)
(q_1) edge node {$0,1$} (q_2)
(q_2) edge node {$0,1$} (q_3)
(q_3) edge node {$0,1$} (q_4)
(q_4) edge node {$0,1$} (q_5);
\endtikzpicture

\item{} i.

\tikzpicture[auto]
\node[state,initial,accepting,initial text=] (q_0) {$q_0$};
\node[state,accepting] (q_1) [right=of q_0] {$q_1$};
\path[->]
(q_0) edge [bend left=10] node {$1$} (q_1)
(q_1) edge [bend left=10] node {$0,1$} (q_0);
\endtikzpicture

\item{7.} e.

\tikzpicture[auto]
\node[state,initial,initial text=] (q_0) {$q_0$};
\node[state] (q_1) [right=of q_0] {$q_1$};
\node[state,accepting] (q_2) [right=of q_1] {$q_1$};
\path[->]
(q_0) edge [loop above] node {$0$} ()
      edge node {$\epsilon$} (q_1)
(q_1) edge [loop above] node {$1$} ()
      edge node {$0$} (q_2)
(q_2) edge [loop above] node {$0$} ();
\endtikzpicture

\item{8.} a.

\tikzpicture[auto]
\node[state,initial,initial text=] (q_0) {$q_0$};
\node[state] (q_1) [right=of q_0] {$q_1$};
\node[state,accepting] (q_2) [right=of q_1] {$q_2$};
\node[state] (q_3) [below=of q_0] {$q_3$};
\node[state] (q_4) [right=of q_3] {$q_4$};
\node[state] (q_5) [right=of q_4] {$q_5$};
\node[state,accepting] (q_6) [right=of q_5] {$q_6$};
\path[->]
(q_0) edge node {$1$} (q_1)
      edge node {$\epsilon$} (q_3)
(q_1) edge [loop above] node {$0,1$} ()
      edge node {$0$} (q_2)
(q_3) edge [loop below] node {$0$} ()
      edge node {$1$} (q_4)
(q_4) edge [loop below] node {$0$} ()
      edge node {$1$} (q_5)
(q_5) edge [loop below] node {$0$} ()
      edge node {$1$} (q_6)
(q_6) edge [loop right] node {$0,1$} ();
\endtikzpicture

\item{9.} a.

\tikzpicture[auto]
\node[state,initial,initial text=] (q_0) {$q_0$};
\node[state] (q_1) [right=of q_0] {$q_1$};
\node[state] (q_2) [right=of q_1] {$q_2$};
\node[state] (q_3) [right=of q_2] {$q_3$};
\node[state] (q_4) [right=of q_3] {$q_4$};
\node[state,accepting] (q_5) [right=of q_4] {$q_5$};
\node[state,accepting] (q_6) [right=of q_5] {$q_6$};
\path[->]
(q_0) edge node {$0,1,\epsilon$} (q_1)
(q_1) edge node {$0,1,\epsilon$} (q_2)
(q_2) edge node {$0,1,\epsilon$} (q_3)
(q_3) edge node {$0,1,\epsilon$} (q_4)
(q_4) edge node {$0,1,\epsilon$} (q_5)
(q_5) edge [bend left=10] node {$1$} (q_6)
(q_6) edge [bend left=10] node {$0,1$} (q_5);
\endtikzpicture

\item{10.} a.

\tikzpicture[auto]
\node[state,initial,initial text=] (q_0) {$q_0$};
\node[state] (q_1) [right=of q_0] {$q_1$};
\node[state] (q_2) [right=of q_1] {$q_2$};
\node[state] (q_3) [right=of q_2] {$q_3$};
\node[state,accepting] (q_4) [right=of q_3] {$q_4$};
\node[state,accepting] (q_5) [below=of q_0] {$q_5$};
\path[->]
(q_0) edge node {$\epsilon$} (q_1)
      edge node {$\epsilon$} (q_5)
(q_1) edge [loop above] node {$0$} ()
      edge node {$1$} (q_2)
(q_2) edge [loop above] node {$0$} ()
      edge node {$1$} (q_3)
(q_3) edge [loop above] node {$0$} ()
      edge node {$1$} (q_4)
(q_4) edge [loop right] node {$0,1$} ();
\endtikzpicture

\goodbreak
\item{31.}

Let $Q_A$, $\delta_A$, $q_{0A}$, and $F_A$ be the
set of states, transition, start state, and set of accepting states
of a deterministic finite automaton $D$ that recognizes $A$.

We construct a nondeterministic finite automaton $N$
that recognizes $A^{\cal R}$.
Let $Q = Q_A \cup \{q_0\}$, where $q_0 \notin Q_A$ be
$Q_A$ with an additional element,
$\delta(q_0, \epsilon) = F$ and
$\delta(q, a) = \{x \in Q : \delta_A(x, a) = q\}$
for $q \in Q$ and $a \in \Sigma$ be
the function that maps $(q, a)$ to the preimage of $q$ under
$f(x) = \delta_A(x, a)$,
$q_0$ be the state mentioned in the definition of $Q$,
and $F = \{q_{0A}\}$ be the singleton set containing the start state of $D$.

We must show that $N$ accepts a word $w^{\cal R}$
if and only if the $D$ accepts $w$.

For the forward direction
(if $D$ accepts $w$, then $N$ accepts $w^{\cal R}$),
we simply notice that $N$ can pass through the same states as $D$
in reverse order.

For the backward direction
(if $N$ accepts $w^{\cal R}$, then $D$ accepts $w$),
assume that $N$ accepts $w^{\cal R}$.
We want to show that $D$ accepts $w$.
Take some path through $N$ that leads to $q_{0A}$.
$D$ must take this path in reverse order and accept $w$.

\item{34.}

\tikzpicture[auto]
\node[state,initial,initial text=] (q_0) {$q_0$};
\node[state,accepting] (q_1) [right=of q_0] {$q_1$};
\path[->]
(q_0) edge [loop above] node {${0 \brack 0}, {1 \brack 1}$} ()
      edge node {$1 \brack 0$} (q_1)
(q_1) edge [loop right] node {${0 \brack 0}, {0 \brack 1}, {1 \brack 0}, {1 \brack 1}$} ();
\endtikzpicture

\item{41.}

Let $Q_A$, $\delta_A$, $q_{0A}$, and $F_A$ be the set of states,
transition function, start state, and set of accepting states
of a deterministic finite automaton $D_A$ that recognizes $A$
and let $Q_B$, $\delta_B$, $q_{0B}$, and $F_B$ be the set of states,
transition function, start state, and set of accepting states
of a deterministic finite automaton $D_B$ that recognizes $B$.

We construct a deterministic finite automaton where
$Q = Q_A \times Q_B \times \{0,1\}$,
$\delta((q_A,q_B,0), a) = (\delta_A(q_A),q_B,1)$ and
$\delta((q_A,q_B,1), a) = (q_A,\delta_B(q_B),0)$,
$q_0 = (q_{0A},q_{0B},0)$, and
$F = F_A \times F_B \times \{0\}$.

This automaton recognizes the perfect shuffle of $A$ and $B$.
\bye
