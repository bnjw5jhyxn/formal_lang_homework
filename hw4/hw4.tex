\def\]{\leavevmode\hbox{\tt\char`\ }} % visible space
\def\acc{{\rm accept}}
\def\rej{{\rm reject}}
\def\abs#1{\vert#1\vert}
\def\angles#1{\langle#1\rangle}
\def\card{{\rm Card}}
\def\inpr#1#2{\langle#1,#2\rangle}
\centerline{Steve Hsu\hfill homework 4}
\item{3.6.}

If there is some $s_k$ such that $M$ loops on $s_k$,
then $M$ will never run on $s_{k+1},\ldots$
and will therefore not output any more strings.
\bigskip
\item{3.8.} b.

Let $Q = \{q_0,q_{s1},q_{r1},q_{rf},q_{s20},q_{s0},q_{r0},q_\acc,q_\rej\}$,
$\Sigma = \{0,1\}$,\hfil\break
$\Gamma = \{0=0_0, 1=1_0, 0_1, 1_1, 0_2, 1_2, 0_3, 1_3, \]\}$,
and $\delta$ be such that
$$\eqalign{
\delta(q_0,0_i) &= (0_2,q_{s1},L)\cr
\delta(q_0,1_i) &= (1_2,q_{s1},L)\cr
\delta(q_0,\]) &= (\],q_\acc,L)\cr
\delta(q_{s1},0_i) &= (0_i,q_{s1},R)\cr
\delta(q_{s1},1_{i \equiv 1 \bmod 2}) &= (1_i,q_{s1},R)\cr
\delta(q_{s1},1_{i \equiv 0 \bmod 2}) &= (1_{i+1},q_{r1},L)\cr
\delta(q_{s1},\]) &= (\],q_{rf},L)\cr
\delta(q_{r1},0_{i \le 1}) &= (0_i,q_{r1},L)\cr
\delta(q_{r1},1_{i \le 1}) &= (1_i,q_{r1},L)\cr
\delta(q_{r1},0_{i \ge 2}) &= (0_i,q_{s20},L)\cr
\delta(q_{r1},1_{i \ge 2}) &= (1_i,q_{s20},L)\cr
\delta(q_{rf},0_1) &= (0_1,q_{rf},L)\cr
\delta(q_{rf},1_{i \le 1}) &= (1_i,q_{rf},L)\cr
\delta(q_{rf},0_{i \equiv 0 \bmod 2}) &= (0_i,q_\rej,L)\cr
\delta(q_{rf},0_3) &= (0_3,q_\acc,L)\cr
\delta(q_{rf},1_{i \ge 2}) &= (1_i,q_\acc,L)\cr
\delta(q_{s20},0_{i \equiv 0 \bmod 2}) &= (0_{i+1},q_{s0},R)\cr
\delta(q_{s20},0_{i \equiv 1 \bmod 2}) &= (0_i,q_{s20},R)\cr
\delta(q_{s20},1_i) &= (1_i,q_{s20},R)\cr
\delta(q_{s20},\]) &= (\],q_\rej,R)\cr
\delta(q_{s0},0_{i \equiv 0 \bmod 2}) &= (0_{i+1},q_{r0},L)\cr
\delta(q_{s0},0_{i \equiv 1 \bmod 2}) &= (0_i,q_{s0},R)\cr
\delta(q_{s0},1_i) &= (1_i,q_{s0},R)\cr
\delta(q_{s0},\]) &= (\],q_\rej,R)\cr
\delta(q_{r0},0_{i \le 1}) &= (0_i,q_{r1},L)\cr
\delta(q_{r0},1_{i \le 1}) &= (1_i,q_{r1},L)\cr
\delta(q_{r0},0_{i \ge 2}) &= (0_i,q_{s1},L)\cr
\delta(q_{r0},1_{i \ge 2}) &= (1_i,q_{s1},L)\cr
}$$
and $\delta(q,a)$ is arbitrary for other values of $(q,a)$.

This Turing machine marks each $1$ and cancels it with two $0$'s,
handling special cases as necessary.
\bigskip
\item{3.11.}

Let $M$ be a Turing machine with a singly infinite tape.
We will construct a Turing machine $M'$ with a doubly infinite tape
that recognizes $L(M)$.

Let $q_S$ and $q_L$ not be in the set of states $Q$ of $M$,
let $q_0$ be the start state of $M$,
and let $\]'$ not be in the tape alphabet $\Gamma$ of $M$.
Let $Q' = Q \cup \{q_S, q_L\}$,
where $q_S$ is the start state of $M'$
and $q_\acc$ and $q_\rej$, the accept and reject states, respectively, of $M$,
are the accept and reject states of $M'$.
The transition function $\delta'$ of $M'$ is identical to
the transition function $\delta$ of $M$ except that
$\delta(q_S,a) = (a,q_L,L)$ for all $a \in \Sigma^\star$,
$\delta(q_L,\]) = (\]',q_0,R)$, and
$\delta(q,\]') = (\]',q,R)$ for all $q \in Q$.
This Turing machine uses $\]'$ to mark the left boundary of the tape
and moves back to the starting position whenever the head moves to the boundary.
\bigskip
\item{3.15.} b.

Let $L_1$ and $L_2$ be languages decided
by deciders $M_1$ and $M_2$, respectively.
Given $w \in \Sigma^\star$, let $x_k$ and $y_k$ be strings
such that $\abs{x_k} = k$ and $x \cdot y = w$.
For each integer $n$ such that $0 \le n \le \abs w$,
run $M_1$ on $x_n$ and $M_2$ on $y_n$.
If both accept, then accept $w$.
If at least one rejects for all $n$, then reject $w$.
\medskip
\item{} e.

Let $L_1$ and $L_2$ be languages decided
by deciders $M_1$ and $M_2$, respectively.
Given $w \in \Sigma^\star$, run $M_1$ and $M_2$ on $w$.
If both accept, then accept $w$.
Otherwise, reject $w$.
\bigskip
\item{3.16.} b.

Let $L_1$ and $L_2$ be languages recognized
by Turing machines $M_1$ and $M_2$, respectively.
Given $w \in \Sigma^\star$, let $x_k$ and $y_k$ be strings
such that $\abs{x_k} = k$ and $x \cdot y = w$.
For each positive integer $i$,
for each integer $n$ such that $0 \le n \le \abs w$,
run $M_1$ on $x_n$ and $M_2$ on $y_n$ for $i$ steps each.
If both accept, then accept $w$.
\medskip
\item{} d.

Let $L_1$ and $L_2$ be languages recognized
by Turing machines $M_1$ and $M_2$, respectively.
Given $w \in \Sigma^\star$, run $M_1$ and $M_2$ on $w$ in parallel.
If both accept, then accept $w$.
Otherwise, reject $w$.
\bigskip
\item{3.18.}

For the forward direction, assume that a language $L$ is decidable.
Let $M$ be a decider for $L$.
For each string $w \in \Sigma^\star$ in standard order, run $M$ on $w$.
If $M$ accepts $w$, output it.

For the backward direction, assume that $L$
is enumerable in a standard order.
A Turing machine $M$ can, on input $w$,
examine the inputs of the enumerator.
If it sees a $w$, it accepts $w$.
If it sees a string after $w$ in the order, it rejects $w$.
\bigskip
\item{4.5.}

Let $s_1,s_2,\ldots$ be such that $\Sigma^\star = \{s_1,s_2,\ldots\}$.
For each positive integer $i$,
for each integer $j$ such that $1 \le j \le i$,
run $M$ on $s_j$ for $i$ steps.
If it accepts, then accept $\angles M$.
\bigskip
\item{4.7.}

Suppose for contradiction that $\cal B$ is countable.
Then we can write ${\cal B} = \{(b_{1n}),(b_{2n}),\ldots\}$.
Take the sequence $(a_n)$ defined by
$$a_k = \cases{
0,&if $b_{kk} = 1$\cr
1,&if $b_{kk} = 0$\cr
}$$
Since $(a_n)$ is an infinite sequence of $0$'s and $1$'s,
it should be in $\cal B$.
This is a contradiction since $(a_n) \ne (b_{kn})$
for all positive integers $k$.
Therefore, $\cal B$ must be uncountable.
\bigskip
\item{4.8.}

\proclaim Lemma. $\card\{(i,j,k) : i + j + k = n\} = {n+2 \choose 2}$
for all natural numbers $n$.

Imagine assigning $n$ indistinguishable items
to three indistinguishable buckets.
One way to make this assignment is to have $n + 2$ objects:
the $n$ items and two dividers.
We pick two of the spots to be dividers, assigning items to the rest.
The items to the left of the first divider are in the first bucket,
those between the dividers are in the second bucket, and
those to the right of the second divider are in the third bucket.
Clearly, there are $n+2 \choose 2$ ways to make the assignment.
Let $i$ be the number of items in the first bucket,
$j$ be the number of items in the second bucket, and
$k$ be the number of items in the third bucket.
\medskip
Since, by the lemma, there are finitely many ordered triples
of natural numbers with sum exactly $n$,
there are also finitely many ordered triples of
natural numbers with sum at most $n$
(${0+2 \choose 2} + {1+2 \choose 2} + \cdots + {n+2 \choose 2}$).
We can therefore list all ordered triples with sum $0$,
then all ordered triples with sum $1$, and so on.
Since every ordered triple of natural numbers has a finite sum,
we will list all ordered triples in this manner.
\bigskip
\item{4.18.}

For the forward direction, assume that $C$ is Turing-recognizable.
Let $M$ be a Turing machine that recognizes $C$.
Let $D$ be the set of strings of the form $\inpr x y$ such that
$M$ accepts $x$ within $n$ steps, where $y$ is a binary representation of $n$.
Since we can accept or reject after running $M$ for at most $n$ steps,
$D$ is decidable.
Clearly, if $x \in C$, then $M$ accepts $x$ in some number of finite steps.
Take $y$ to be the binary representation of this number of steps.
Therefore, there is string $y$ such that $\inpr x y \in D$.

For the backward direction, assume that $D$ is decidable
and let $C = \{x : (\exists y) \inpr x y \in D\}$.
Let $M$ be a decider for $D$.
Given an input $x$,
for all $y \in \Sigma^\star$, run $D$ on $\inpr x y$.
If it accepts one, accept $x$.
We have a Turing recognizer for $C$.
\bigskip
\item{4.20.}

Let $M_1$ be a recognizer for $\overline A$ and $M_2$ be a recognizer for $\overline B$.
Consider the Turing machine $D$ that, on input $w$,
runs $M_1$ and $M_2$ on $w$ in parallel.
At each step, if $M_2$ accepts, it accepts $w$.
If $M_1$ accepts and $M_2$ does not accept, it rejects $w$.
Since $A$ and $B$ are disjoint, $\overline A \cup \overline B = \Sigma^\star$.
Therefore, at least one of $M_1$ and $M_2$ accepts any string $w$
and $D$ is a decider.
If $w \in A$, then $w \notin \overline A$ and $w \in \overline B$.
Therefore, $M_2$ will accept $w$ and $M_1$ will not.
Hence $A \subseteq L(D)$.
Similarly, $B \subseteq \overline{L(D)}$.
Consequently, $L(D)$ separates $A$ and $B$.
\bye 
