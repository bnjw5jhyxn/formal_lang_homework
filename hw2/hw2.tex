\def\len#1{\vert{#1}\vert}
\input miniltx
\input graphicx.sty
\input tikz.tex
\usetikzlibrary{automata,positioning}
\centerline{Steve Hsu\hfill homework 2}
\item{16.} b.

\tikzpicture[auto]
\node[state,accepting] (123) {$\{1,2,3\}$};
\node[state,accepting] (23) [right=of 123] {$\{2,3\}$};
\node[state] (13) [right=of 23] {$\{1,3\}$};
\node[state,accepting,initial,initial text=] (12) [below=of 123] {$\{1,2\}$};
\node[state,accepting] (2) [right=of 12] {$\{2\}$};
\node[state] (3) [right=of 2] {$\{3\}$};
\node[state] (e) [below=of 2] {$\emptyset$};
\node[state] (1) [right=of e] {$\{1\}$};
\path[->]
(e) edge [loop below] node {$a,b$} ()
(1) edge node {$a$} (3)
    edge node {$b$} (e)
(2) edge node {$a$} (12)
    edge node {$b$} (e)
(3) edge node {$a$} (2)
    edge node {$b$} (23)
(12) edge node {$a$} (123)
     edge node {$b$} (e)
(13) edge node {$a,b$} (23)
(23) edge node {$a$} (12)
     edge [loop above] node {$b$} ()
(123) edge [loop above] node {$a$} ()
      edge node {$b$} (23);
\endtikzpicture
\bigskip
\item{17.} a.

\tikzpicture[auto]
\node[state,accepting,initial,initial text=] (q0) {$q_0$};
\node[state] (q1) [below=of q0] {$q_1$};
\node[state] (q2) [left=of q1] {$q_2$};
\node[state] (q3) [right=of q1] {$q_3$};
\path[->]
(q0) edge node {$0$} (q1)
(q1) edge node {$0$} (q2)
     edge [swap] node {$1$} (q3)
(q2) edge node {$1$} (q0)
(q3) edge [swap] node {$0,\epsilon$} (q0);
\endtikzpicture
\bigskip
\item{20.} e.
\item{} members: ``aba'', ``aaaabbaabb''
\item{} not members: ``$\epsilon$'', ``aaaabb''
\bigskip
\item{21.} b. $\epsilon \cup ((a \cup b) a^\star b
((aa \cup ab \cup b) a^\star b)^\star (a \cup \epsilon))$
\bigskip
\item{29.} b.

Assume for contradiction that $A_2$ is regular.
Then there is a natural number $p$ as described in the pumping lemma.
Consider the string $0^p 1 0^p 1 0^p 1$,
which is in $A_2$ and has length greater than $p$.
We can partition this string into $x$, $y$, and $z$
as described in the pumping lemma.
Since the $\len x + \len y \le p$,
$y$ must consist entirely of zeroes.
Taking $x y^2 z$, we observe that the string is now
$0^{p + \len y} 1 0^p 1 0^p 1$, which is clearly not in $A_2$.
This is a contradiction since the pumping lemma says that
it should be in $A_2$.
\bigskip
\item{37.}

Observe that there at most $n$ distinct powers of $2$ modulo $n$.
We can therefore construct an automaton with $n^2$ states,
which keep track of the current total modulo $n$
and the place value of the current digit modulo $2$.
The start state corresponds to a place value of $1$
and a total of $0$ modulo $n$.
All states representing a total of $0$ modulo $n$ are accepting states.
\bigskip
\item{43.}

Let $A$ be a regular language and
let $D$ be a deterministic finite automaton recognizing $A$.
We will construct a nondeterministic finite automaton $N$
recognizing $DROP-OUT(A)$.
If $D$ has $q$ states, then $N$ will have $3n$ states.
They will be in three groups of $n$, each of which is identical to $D$.
We will call these groups layer $0$, layer $1$, and layer $2$.
In addition to the transitions present in $D$,
there will also be epsilon transitions between layers.
For example, if $D$ has a transition on symbol $a$
from state $q$ to state $r$, then $N$ will have an epsilon transition
from state $q$ on layer $x$ to state $r$ on layer $x + 1$.
The accepting states will be those on layer $1$ corresponding
to the accepting states of $D$.
\bigskip
\item{70.}

Let $A$ and $B$ be regular languages recognized by
deterministic finite automata $D$ and $E$, respectively.
We will first construct a nondeterministic finite automaton $F$ that
recognizes all strings containing and element of $B$ as a substring.
$F$ is similar to $E$, but has a new start state with a self-loop
on all elements of $\Sigma$ and an epsilon transition
from the new start state to the start state of $E$.
$F$ also has epsilon transitions
from the accepting states of $E$ to a new accepting state,
which also has a self-loop on all elements of $\Sigma$.
Clearly $L(F)$ is regular and there is consequently
a deterministic finite automaton $G$
recognizing the complement of $L(F)$.

We will construct a deterministic finite automaton $H$
recognizing $A$ avoids $B$.
The states of $H$ are pairs of states from $D$ and $G$.
Its starting state is the starting state of $D$
paired with the starting state of $G$.
Its transition function applies the transition
functions of $D$ and $G$ to the appropriate components of the pair.
Its accepting states are pairs where both states are accepting.
\bigskip
\item{71.} a.

$A$ can be specified by the regular expression $0 (0 \cup 1)^\star 0$.
\medskip
\item{} b.

Assume for contradiction that $B$ is regular.
Then there is a natural number $p$ as described in the pumping lemma.
Take the string $0^p 1 0^p$, which is in $B$ and has length greater than $p$.
We can partition this string into $x$, $y$, and $z$
as described in the pumping lemma.
Since $\len x + \len y \le p$, $y$ consists entirely of zeroes.
Observe that $x y^2 z = 0^{p + \len y} 1 0^p$.
For this string to be in $B$, $k$ must be exactly $p + \len y$,
but there aren't enough zeroes at the end
and $x y^2 z$ cannot be in $B$.
This is a contradiction because the pumping lemma says that
$x y^2 z$ should be in $B$.
\bye
