\def\len#1{\vert{#1}\vert}
\input miniltx
\input graphicx.sty
\input tikz.tex
\centerline{Steve Hsu\hfill homework 3}
\item{1.} a. $E \Rightarrow T \Rightarrow F \Rightarrow a$

\tikzpicture[level distance=0.3in]
\node {E}
	child {node {T}
		child {node {F}
			child {node {a}}
		}
	}
;
\endtikzpicture
\medskip
\item{} b. $E \Rightarrow E+T \Rightarrow T+T \Rightarrow F+F \Rightarrow a+a$

\tikzpicture[level distance=0.3in]
\node {E}
	child {node {E}
		child {node {T}
			child {node {F}
				child {node {a}}
			}
		}
	}
	child {node {$+$}}
	child {node {T}
		child {node {F}
			child {node {a}}
		}
	}
;
\endtikzpicture
\medskip
\item{} c. $E \Rightarrow E+T \Rightarrow E+T+T \Rightarrow T+T+T \Rightarrow F+F+F \Rightarrow a+a+a$

\tikzpicture[level distance=0.3in]
\node {E}
	child {node {E}
		child {node {E}
			child {node {T}
				child {node {F}
					child {node {a}}
				}
			}
		}
		child {node {$+$}}
		child {node {T}
			child {node {F}
				child {node {a}}
			}
		}
	}
	child {node {$+$}}
	child {node {T}
		child {node {F}
			child {node {a}}
		}
	}
;
\endtikzpicture
\medskip
\item{} d. $E \Rightarrow T \Rightarrow F \Rightarrow (E) \Rightarrow (T)
\Rightarrow (F) \Rightarrow ((E)) \Rightarrow ((T)) \Rightarrow ((F)) \Rightarrow ((a))$

\tikzpicture[level distance=0.3in]
\node {E}
	child {node {T}
		child {node {F}
			child {node {(}}
			child {node {E}
				child {node {T}
					child {node {F}
						child {node {(}}
						child {node {E}
							child {node {T}
								child {node {F}
									child {node {a}}
								}
							}
						}
						child {node {)}}
					}
				}
			}
			child {node {)}}
		}
	}
;
\endtikzpicture
\bigskip
\item{4.} c.
$$\eqalign{
S &\rightarrow TB\cr
T &\rightarrow TBB \vert \epsilon\cr
B &\rightarrow 0 \vert 1\cr
}$$
\medskip
\item{} e.
$$S \rightarrow 0S0 \vert 1S1 \vert \epsilon$$
\bigskip
\item{9.}
$$\eqalign{
S &\rightarrow AR \vert LC\cr
L &\rightarrow aLb \vert \epsilon\cr
R &\rightarrow bRc \vert \epsilon\cr
A &\rightarrow Aa \vert \epsilon\cr
C &\rightarrow Cc \vert \epsilon\cr
}$$
This grammar is ambiguous because strings of the form $a^n b^n c^n$,
where $n \ge 0$, can be derived from either $AR$ or $LC$.
\bigskip
\item{10.}

First, the automaton goes from a start state to one of two options:
either it can match the $a$'s against the $b$'s
or it can match the $b$'s against the $c$'s.
If it matches the $a$'s against the $b$'s,
then it pushes a symbol onto the stack for each $a$
and pops one for each $b$, then ignores the $c$'s.
If it matches the $b$'s against the $c$'s,
then it first ignores the $a$'s,
pushes a symbol onto the stack for each $b$,
and pops a symbol for each $c$.
\bigskip
\item{6.} b.

There two types of strings in the complement of $a^n b^n$:
there are those where all $a$'s come before all $b$'s,
but the number of $a$'s and the number of $b$'s are not equal,
and those where there is a $b$ before an $a$.
This observation motivates the following grammar:
$$\eqalign{
S &\rightarrow AE \vert EB \vert M\cr
E &\rightarrow aEb \vert \epsilon\cr
M &\rightarrow XM \vert MX \vert ba\cr
X &\rightarrow a \vert b\cr
A &\rightarrow Aa \vert a\cr
B &\rightarrow Bb \vert b\cr
}$$
\bigskip
\item{13.}

$L(G)$ is the language of strings over the alphabet $\Sigma = \{0,\#\}$
that either contain two hashes or are of the form $0^n \# 0^{2n}$,
where $n \ge 0$.
\bigskip
\item{14.}

$$\eqalign{
S &\rightarrow B A_1 \vert B_1 B_1 \vert \epsilon\cr
A &\rightarrow B A_1 \vert B_1 B_1\cr
A_1 &\rightarrow AB\cr
B &\rightarrow B_1 B_1\cr
B_1 &\rightarrow 0\cr
}$$
\bigskip
\item{16.}

Let $A$ and $B$ be context-free languages generated by grammars with
with starting symbols $S_A$ and $S_B$, respectively
and no nonterminals with the same name.
Then $A \cup B$, the union of $A$ and $B$,
can be generated by the grammar containing the rule
$S \rightarrow S_A \vert S_B$, along with the rules of $A$ and $B$.
$A \cdot B$, the concatenation of $A$ and $B$,
can be generated by the grammar containing the rule
$S \rightarrow S_A S_B$, along with the rules of $A$ and $B$.
$A^\star$, the Kleene star of $A$, can be generated by the grammar containing
$S \rightarrow S S_A \vert \epsilon$, along with the rules of $A$.
\bigskip
\item{26.}

We can first perform $n - 1$ steps to convert the starting symbol into $n$
nonterminals, each of which will expand in $1$ step to a terminal in $w$.
\bigskip
\item{30.} a.

Suppose for contradiction that $L = \{0^n 1^n 0^n 1^n : n \ge 0\}$
is context-free.
Let $p$ be the pumping length as described in the pumping lemma.
Take $w = 0^p 1^p 0^p 1^p$, which has length $4p$ and is in $L$.
Then, by the pumping lemma, we can partition $w$ into
$uvxyz$, where $\len{vxy} \le p$.
We have two cases: first, $v$ or $y$ contains a $0$;
and second, $v$ or $y$ contains a $1$.
In the first case, observe that $vxy$ cannot reach both regions of $0$'s.
Therefore, $w' = u v^2 x y^2 z$ has more $0$'s in one region
than the other and it is consequently not in $L$.
This is a contradiction because
the pumping lemma says that $w'$ should be in $L$.
The argument for the second case is similar.
\bigskip
\item{31.}

Suppose for contradiction that $B$ is context-free.
Let $p$ be the pumping length as described in the pumping lemma.
Take $w = 0^p 1^{2p} 0^p$, which has length $4p$ and is in $B$.
Then, by the pumping lemma, we can partition $w$ into
$uvxyz$, where $\len{vxy} \le p$.
We have two cases: first, $vxy$ consists entirely of $1$'s,
and second, $vxy$ contains at least one $0$.
In the first case, $w' = u v^2 x y^2 z$ has more $1$'s than $0$'s
and is consequently not in $B$.
This is a contradiction because the pumping lemma says that
$w'$ should be in $B$.
In the second case, since $vxy$ cannot reach both regions of $0$'s,
$w' = u v^2 x y^2 z$ has more $0$'s in one region than the other
and is therefore not a palindrome, so it cannot be in $B$.
This is a contradiction because the pumping lemma says thet $w'$
should be in $B$.
\bye
